\documentclass[12pt,journal,letterpaper,compsoc]{IEEEtran}
\usepackage{cite}
\usepackage{url}
\usepackage{times}
\usepackage{graphicx}
\usepackage{url}
\usepackage{clrscode}
\usepackage{tabularx}
\usepackage{amsmath}
\usepackage{array}
\usepackage{color}
\usepackage{balance}

\begin{document}

\title{}

\author{Karim Ali\\
\{karim\}@cs.uwaterloo.ca \\
David R. Cheriton School of Computer Science\\
University of Waterloo\\
}

% The paper headers
\markboth{CS886 Project Report - Active Contour without Edges}%
{Shell \MakeLowercase{\textit{et al.}}: Bare Demo of IEEEtran.cls for Computer Society Journals}

\IEEEcompsoctitleabstractindextext{%
\begin{abstract}
%\boldmath
The abstract goes here.
\end{abstract}
}

\maketitle

% # Introduction
%     * What is the problem?
%     * Why is it an important problem?
% 
% # Techniques to tackle the problem
%     * Brief survey of previous work concerning this problem (i.e., the 4-8 papers that you read)
%     * Brief description of the techniques chosen and why
% 
% # Empirical evaluation
%     * Compare empirically the techniques for complexity, performance, ease of use, etc.
% 
% # Conclusion:
%     * What is the best technique?
%     * Is any technique good enough to declare the problem solved?
%     * What future research do you recommend?

\section{Introduction}
\label{sec:intro}

\section{Background}
\label{sec:bg}

\begin{figure}[t!]
\centering
\includegraphics[width=6cm]{fitting.png}
\caption{Chan-Vese Curve Fitting}
\label{fig:fitting}
\end{figure}

\section{Implementation}
\label{sec:implementation}

\section{Experimental Results}
\label{sec:results}

\begin{table*}[h!t!]
\centering
\begin{tabular}{ c | c |c}
\textbf{Criteria} & \textbf{Goal} & \textbf{Applies to Model}\\
\hline\hline
Detecting Boundaries & Ability to correctly detect object boundaries of simple objects& Both Models\\
\hline
\end{tabular}
\centering
\caption{Evaluation Criteria}
\label{tab:criteria}
\end{table*}

\section{Difficulties and Discussion}
\label{sec:difficulties}

\section{Conclusion}
\label{sec:conc}
The conclusion goes here.


\bibliographystyle{IEEEtran}
\bibliography{references}
% insert where needed to balance the two columns on the last page with
% biographies
%\newpage

\end{document}