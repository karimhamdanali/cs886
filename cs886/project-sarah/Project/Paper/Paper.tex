
\documentclass{article}

% For figures
\usepackage{graphicx} 

% For citations
\usepackage{natbib}

% For algorithms
\usepackage{algorithm}
\usepackage{algorithmic}

% As of 2010, we use the hyperref package to produce hyperlinks in the
% resulting PDF.  If this breaks your system, please commend out the
% following usepackage line and replace \usepackage{icml2010} with
% \usepackage[nohyperref]{icml2010} above.
\usepackage{hyperref}

% Packages hyperref and algorithmic misbehave sometimes.  We can fix
% this with the following command.
\newcommand{\theHalgorithm}{\arabic{algorithm}}
\usepackage[accepted]{icml2010}


% The \icmltitle you define below is probably too long as a header.
% Therefore, a short form for the running title is supplied here:
\icmltitlerunning{Title}

\begin{document} 

\twocolumn[
\icmltitle{Using Bayesian networks to predict changes}


\icmlauthor{Sarah Nadi}{snadi@uwaterloo.ca}
\icmladdress{University of Waterloo,
            Waterloo, ON, Canada}

\vskip 0.3in
]

\begin{abstract} 
STILLLL
\end{abstract} 

\section{Introduction}
\label{intro}

Changes 

\begin{algorithm}[tb]
   \caption{Bubble Sort}
   \label{alg:example}
\begin{algorithmic}
   \STATE {\bfseries Input:} data $x_i$, size $m$
   \REPEAT
   \STATE Initialize $noChange = true$.
   \FOR{$i=1$ {\bfseries to} $m-1$}
   \IF{$x_i > x_{i+1}$} 
   \STATE Swap $x_i$ and $x_{i+1}$
   \STATE $noChange = false$
   \ENDIF
   \ENDFOR
   \UNTIL{$noChange$ is $true$}
\end{algorithmic}
\end{algorithm}
 

\begin{table}[t]
\caption{Classification accuracies for naive Bayes and flexible 
Bayes on various data sets.}
\label{sample-table}
\vskip 0.15in
\begin{center}
\begin{small}
\begin{sc}
\begin{tabular}{lcccr}
\hline
\abovespace\belowspace
Data set & Naive & Flexible & Better? \\
\hline
\abovespace
Breast    & 95.9$\pm$ 0.2& 96.7$\pm$ 0.2& $\surd$ \\
Cleveland & 83.3$\pm$ 0.6& 80.0$\pm$ 0.6& $\times$\\
Glass2    & 61.9$\pm$ 1.4& 83.8$\pm$ 0.7& $\surd$ \\
Credit    & 74.8$\pm$ 0.5& 78.3$\pm$ 0.6&         \\
Horse     & 73.3$\pm$ 0.9& 69.7$\pm$ 1.0& $\times$\\
Meta      & 67.1$\pm$ 0.6& 76.5$\pm$ 0.5& $\surd$ \\
Pima      & 75.1$\pm$ 0.6& 73.9$\pm$ 0.5&         \\
\belowspace
Vehicle   & 44.9$\pm$ 0.6& 61.5$\pm$ 0.4& $\surd$ \\
\hline
\end{tabular}
\end{sc}
\end{small}
\end{center}
\vskip -0.1in
\end{table}



\section{Related Work}
\label{rel-work}

Mirarab et al.~\cite{mirarab2007} investigate the same problem as our work. However, their work is on the level of source code changes. They build three
different Bayesian Networks, one that is based on package and class dependency information (static relationships), one which is dependent on historical
co-changes, and one which uses both. For the first graph, the initial structure is essentially ``given'' according to the static dependencies, and then the CPTs
are learnt using the importance sampling algorithm proposed by Changhe and Marek~\cite{yuan2003importance}. The way static dependencies are defined in their
case is specific to Java. The third one is essentially the first graph, but updated using the historic change information according to the Expectation
Maximization (EM) algorithm~\cite{dempster1977maximum}. The second was solely based on historic information where the network is build using a greedy structure
 learning algorithm~ \cite{friedman1996learning}. They did some preprocessing to their data such as filtering out large changes (with more than 30 elements
changed at once) since this was probably an insignificant change.

Zhou et al.~\cite{zhou2008} try to answer a slightly different problem. They do not only look at the probability of other elements changing given a specific
element, they also add features such as authors, change significance levels etc. and try to predict if two elements are co-changes or not accordingly. Thus,
their problem is more of a classification problem where given two elements, and some observed features they try to determine the class as co_changes or not.
They use the K2 algorithm proposed by Cooper et. al~\cite{cooper1992bayesian} to estimate the structure of the Bayesian network, and use the SimpleEstimator
algorithm built in WEKA~\cite{witten2005data}.

\section{Building the model}
\label{sec:model}

\section{Experimental Work}
\label{sec:exp}

\section{Conclusion}
\label{concl}


\bibliography{references}
\bibliographystyle{icml2010}

\end{document} 


